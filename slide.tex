% !TeX root = slide.tex
\documentclass{beamer}

% 核心包
\usepackage{ctex}
\usepackage[T1]{fontenc}
\usepackage{hyperref}

% 加载OUC样式
\usepackage{OUC}

% 数学和图形包
\usepackage{amsmath, amssymb}
\usepackage{graphicx}
\usepackage{multicol}
\usepackage{booktabs}

%下划线
\usepackage[normalem]{ulem}

% 元数据设置
\title{基于海洋主题的演示文稿}
\subtitle{中国海洋大学Beamer模板示例}
\author{xianrui5891}
\institute{中国海洋大学崇本学院}
\date{2025年8月5日}


\begin{document}

% 生成封面
\maketitle

% 目录页
\tocbackground
\begin{frame}{目录}
    \begin{multicols}{2}
        \small
        \tableofcontents[sectionstyle=show/shaded,subsectionstyle=show/shaded/hide,subsubsectionstyle=show/shaded/hide]
    \end{multicols}
\end{frame}
\contentbackground

% 第一部分
\section{模板特色介绍}

\begin{frame}{海洋色系设计}
    \begin{block}{主要特点}
        \begin{itemize}
            \item 采用\oucemph{海洋蓝色系}作为主色调
            \item 半透明边框设计,保持背景图片完整性
            \item 4:3比例适配,专为学术演示优化
        \end{itemize}
    \end{block}
    
    \begin{exampleblock}{示例}
        这是一个示例环境,展示\oceanblue{海洋蓝}色彩搭配效果。
    \end{exampleblock}
\end{frame}

\begin{frame}{字体配置}
    \begin{columns}
        \begin{column}{0.5\textwidth}
            \begin{block}{中文字体}
                \begin{itemize}
                    \item 正文:楷体
                    \item 数学:楷体
                \end{itemize}
            \end{block}
        \end{column}
        \begin{column}{0.5\textwidth}
            \begin{exampleblock}{English Fonts}
                \begin{itemize}
                    \item Text: Times New Roman
                    \item Math: Times New Roman
                \end{itemize}
            \end{exampleblock}
        \end{column}
    \end{columns}
    
    \vskip 1cm
    数学公式示例:$E = mc^2$, $\int_0^\infty e^{-x} dx = 1$
\end{frame}

% 第二部分
\section{功能演示}

\begin{frame}{定理和证明环境}
    \begin{theorem}[勾股定理]
        在直角三角形中,两直角边的平方和等于斜边的平方。
        $$a^2 + b^2 = c^2$$
    \end{theorem}
    
    \begin{proof}
        这是一个经典的几何定理,可以通过多种方法证明...
    \end{proof}
\end{frame}

\begin{frame}[fragile]{代码展示}
    \begin{code}[Python代码示例]{python}
import numpy as np
import matplotlib.pyplot as plt

x = np.linspace(0, 4*np.pi, 100)
y = np.sin(x) + 0.5*np.sin(3*x)

plt.plot(x, y, 'b-', linewidth=2, label='Ocean Wave')
plt.xlabel('Distance')
plt.ylabel('Height')
plt.title('Ocean Wave Simulation')
plt.legend()
plt.show()
    \end{code}
\end{frame}

\begin{frame}{表格展示}
    \begin{table}
        \centering
        \caption{\deepocean{海洋数据统计表}}
        \begin{tabular}{@{}lcc@{}}
            \toprule
            \textbf{海域} & \textbf{平均深度(m)} & \textbf{最大深度(m)} \\
            \midrule
            渤海 & 18 & 86 \\
            黄海 & 44 & 152 \\
            东海 & 349 & 2,719 \\
            南海 & 1,212 & 5,559 \\
            \bottomrule
        \end{tabular}
    \end{table}
\end{frame}

\begin{frame}{多种Block样式展示}
    \begin{block}{标准Block}
        这是标准的block环境,使用深蓝色标题。
    \end{block}
    
    \begin{exampleblock}{示例Block}
        这是exampleblock环境,使用淡绿色系配色方案。
    \end{exampleblock}
    
    \begin{alertblock}{警告Block}
        这是alertblock环境,使用粉红系配色方案。
    \end{alertblock}
\end{frame}

% 第三部分
\section{总结}

\begin{frame}{模板优势}
    \begin{alertblock}{设计亮点}
        \begin{enumerate}
            \item \oucalert{背景图完整保留}:无额外图案添加
            \item \oucalert{色彩协调统一}:基于海洋主题色系
            \item \oucalert{布局合理优化}:左侧内容,底部导航
            \item \oucalert{字体搭配专业}:中英文分别优化
            \item \oucalert{半透明Block}:多种颜色配色方案
        \end{enumerate}
    \end{alertblock}
    
    \vskip 0.5cm
    \centerline{\Large \lightocean{感谢使用中国海洋大学Beamer模板!}}
\end{frame}

\begin{frame}{}
    \centering
    \vspace{2cm}
    
    {\Huge \deepocean{谢谢聆听!}}
    
    \vspace{1cm}
    
    {\Large \oceanblue{欢迎提出宝贵意见和建议}}
    
    \vspace{1cm}
    
    {\normalsize 联系方式:zhuxianrui@stu.ouc.edu.cn}
\end{frame}

\end{document}